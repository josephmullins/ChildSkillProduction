\documentclass{article}
\usepackage{booktabs,caption}
\usepackage{geometry}
\usepackage{color}
\usepackage{pdflscape}

\title{Summary of GMM Estimates}
\author{your old pal Jo}

\begin{document}

\maketitle

\begin{table}\footnotesize{Joint GMM Estimation of Production and Demand - Preferred Specification}
    \begin{center}
        \begin{tabular}{lcccccccc}\\\toprule
 & \multicolumn{2}{c}{$\epsilon_{\tau,g}$} & \multicolumn{2}{c}{$\epsilon_{x,H}$} & \multicolumn{2}{c}{$\delta_{1}$} & \multicolumn{2}{c}{$\delta_{2}$} \\
& $(\kappa=0)$ & $(\kappa=1)$ & $(\kappa=0)$ & $(\kappa=1)$ & $(\kappa=0)$ & $(\kappa=1)$ & $(\kappa=0)$ & $(\kappa=1)$ \\\cmidrule(r){2-3}\cmidrule(r){4-5}\cmidrule(r){6-7}\cmidrule(r){8-9}
&$0.21^{+}$&$0.21^{+}$&$0.49^{+}$&$0.50^{+}$&0.11&0.13&0.87&0.87\\
&(0.05)&(0.05)&(0.09)&(0.09)&(0.04)&(0.04)&(0.01)&(0.02)\\
&&&&&&&&\\
 & \multicolumn{2}{c}{$\tilde{\phi}_{m}$: Mother's Time} & \multicolumn{2}{c}{$\tilde{\phi}_{f}$: Father's Time} & \multicolumn{2}{c}{$\tilde{\phi}_{x}$: Childcare} & \multicolumn{2}{c}{$\phi_{\theta}$: TFP} \\
& $(\kappa=0)$ & $(\kappa=1)$ & $(\kappa=0)$ & $(\kappa=1)$ & $(\kappa=0)$ & $(\kappa=1)$ & $(\kappa=0)$ & $(\kappa=1)$ \\\cmidrule(r){2-3}\cmidrule(r){4-5}\cmidrule(r){6-7}\cmidrule(r){8-9}
Constant&$7.95$&$7.86$&$4.39$&$4.34$&$-1.17$&$-1.18$&2.44&2.29\\
&(1.72)&(1.69)&(1.21)&(1.20)&(0.41)&(0.41)&(0.42)&(0.47)\\
Single&$0.17$&$0.18$&-&-&$0.63$&$0.63$&-0.18&-0.22\\
&(0.34)&(0.34)&-&-&(0.21)&(0.21)&(0.07)&(0.08)\\
Type 2&$-1.08$&$-1.08$&-&-&$-0.02$&$-0.02$&0.18&0.13\\
&(0.53)&(0.53)&-&-&(0.30)&(0.30)&(0.09)&(0.10)\\
Type 3&$-2.24$&$-2.24$&-&-&$-0.02$&$-0.02$&-0.10&-0.18\\
&(0.83)&(0.82)&-&-&(0.31)&(0.31)&(0.12)&(0.14)\\
Mother some coll.&$-0.51$&$-0.50$&-&-&$0.08$&$0.08$&0.10&0.06\\
&(0.42)&(0.42)&-&-&(0.20)&(0.20)&(0.07)&(0.08)\\
Mother coll+&$-1.62$&$-1.58$&-&-&$-0.12$&$-0.12$&-0.03&-0.07\\
&(0.67)&(0.66)&-&-&(0.20)&(0.20)&(0.09)&(0.11)\\
Child's age&$-0.56$&$-0.55$&$-0.48^{+}$&$-0.47$&$-0.07$&$-0.07$&-0.19&-0.19\\
&(0.16)&(0.15)&(0.16)&(0.16)&(0.03)&(0.03)&(0.03)&(0.04)\\
Num. of children 0-5&$0.43$&$0.42$&$0.54$&$0.54$&$0.05$&$0.05$&0.07&0.06\\
&(0.27)&(0.27)&(0.38)&(0.38)&(0.12)&(0.12)&(0.05)&(0.05)\\
Father some coll.&-&-&$-1.15$&$-1.15$&$-0.04$&$-0.04$&-0.02&-0.03\\
&-&-&(0.69)&(0.68)&(0.25)&(0.25)&(0.07)&(0.08)\\
Father coll+&-&-&$-0.88$&$-0.87$&$-0.58$&$-0.58$&0.20&0.17\\
&-&-&(0.62)&(0.62)&(0.24)&(0.23)&(0.07)&(0.08)\\
Year = 2002&-&-&-&-&-&-&-0.39&-0.36\\
&-&-&-&-&-&-&(0.05)&(0.06)\\
\\
\bottomrule\end{tabular}
    \end{center}
\end{table}

\begin{table}\footnotesize{Joint GMM Estimation of Demand - Preferred Specification}
    \begin{center}
        \begin{tabular}{lcccccccc}\\\toprule
 & \multicolumn{2}{c}{$\epsilon_{\tau,g}$} & \multicolumn{2}{c}{$\epsilon_{x,H}$} & \multicolumn{2}{c}{Correl. residuals}\\
&(1)&(2)&(1)&(2)&(1)&(2)\\\cmidrule(r){2-3}\cmidrule(r){4-5}\cmidrule(r){6-7}
&0.20&0.37&0.52&0.77&0.88&0.88&&\\
&(0.05)&(0.17)&(0.09)&(0.09)&&&&\\
&&&&&&\\
 & \multicolumn{2}{c}{$\tilde{\phi}_{m}$: Mother's Time} & \multicolumn{2}{c}{$\tilde{\phi}_{f}$: Father's Time} & \multicolumn{2}{c}{$\tilde{\phi}_{x}$: Childcare}\\
&(1)&(2)&(1)&(2)&(1)&(2)\\\cmidrule(r){2-3}\cmidrule(r){4-5}\cmidrule(r){6-7}
Const.&8.32&5.54&4.10&3.35&-1.19&-1.45\\
&(1.95)&(1.67)&(1.27)&(0.76)&(0.40)&(0.28)\\
Single&0.29&0.11&-&-&0.62&0.63\\
&(0.38)&(0.21)&-&-&(0.21)&(0.14)\\
Type 2&-1.14&-0.49&-&-&0.03&0.00\\
&(0.59)&(0.46)&-&-&(0.29)&(0.20)\\
Type 3&-2.46&-1.08&-&-&-0.04&-0.10\\
&(0.95)&(0.86)&-&-&(0.30)&(0.21)\\
Mother: Some College&-0.44&-0.13&-&-&-0.01&-0.06\\
&(0.45)&(0.28)&-&-&(0.19)&(0.13)\\
Mother: College+&-1.79&-0.76&-&-&-0.20&-0.28\\
&(0.76)&(0.65)&-&-&(0.19)&(0.13)\\
Child Age&-0.60&-0.34&-0.48&-0.24&-0.07&-0.04\\
&(0.18)&(0.15)&(0.18)&(0.14)&(0.03)&(0.02)\\
Num. Children 0-5&0.34&0.16&0.59&0.29&0.09&0.08\\
&(0.29)&(0.18)&(0.41)&(0.26)&(0.12)&(0.08)\\
Father: College+&-&-&-1.08&-0.41&0.06&0.01\\
&-&-&(0.73)&(0.45)&(0.25)&(0.17)\\
Father: Some College&-&-&-0.83&-0.22&-0.53&-0.41\\
&-&-&(0.66)&(0.44)&(0.23)&(0.16)\\
\\
\bottomrule\end{tabular}
    \end{center}
\end{table}

\begin{landscape}
    \begin{table}\footnotesize\caption{\label{res2}Joint GMM Estimation - Fully Restricted Case, No Binding Constraints}
        \begin{center}
            \begin{tabular}{lcccccccccccccccc}\\\toprule
 & \multicolumn{4}{c}{$\epsilon_{\tau,g}$} & \multicolumn{4}{c}{$\epsilon_{x,H}$} & \multicolumn{4}{c}{$\delta_{1}$} & \multicolumn{4}{c}{$\delta_{2}$} \\
&(1)&(2)&(3)&(4)&(1)&(2)&(3)&(4)&(1)&(2)&(3)&(4)&(1)&(2)&(3)&(4)\\\cmidrule(r){2-5}\cmidrule(r){6-9}\cmidrule(r){10-13}\cmidrule(r){14-17}
&$0.32$&$0.26$&$0.20$&$0.18$&$0.52$&$0.44$&$0.49$&$0.50$&0.06&0.08&0.07&0.07&0.93&0.93&0.93&0.94\\
&(0.05)&(0.05)&(0.05)&(0.05)&(0.08)&(0.08)&(0.08)&(0.08)&(0.04)&(0.04)&(0.04)&(0.04)&(0.01)&(0.01)&(0.01)&(0.01)\\
&&&&&&&&&&&&&&&&\\
 & \multicolumn{4}{c}{$\tilde{\phi}_{m}$: Mother's Time} & \multicolumn{4}{c}{$\tilde{\phi}_{f}$: Father's Time} & \multicolumn{4}{c}{$\tilde{\phi}_{x}$: Childcare} & \multicolumn{4}{c}{$\phi_{\theta}$: TFP} \\
&(1)&(2)&(3)&(4)&(1)&(2)&(3)&(4)&(1)&(2)&(3)&(4)&(1)&(2)&(3)&(4)\\\cmidrule(r){2-5}\cmidrule(r){6-9}\cmidrule(r){10-13}\cmidrule(r){14-17}
Constant&$5.07$&$6.33$&$8.34$&$13.12$&$3.27^{+}$&$3.50^{+}$&$4.09$&$4.16$&$-1.17^{+}$&$-1.21^{+}$&$-1.19^{+}$&$-1.46^{+}$&-0.78&-1.16&-1.05&-0.49\\
&(0.68)&(1.02)&(1.98)&(3.94)&(0.74)&(0.90)&(1.27)&(1.41)&(0.32)&(0.44)&(0.41)&(0.61)&(0.46)&(0.49)&(0.40)&(0.29)\\
Single&$0.13$&$0.11^{+}$&$-0.02$&$0.08^{+}$&-&-&-&-&$0.52^{+}$&$0.52$&$0.57$&$0.60^{+}$&-0.08&-0.07&-0.07&-0.06\\
&(0.24)&(0.28)&(0.37)&(0.41)&-&-&-&-&(0.20)&(0.24)&(0.21)&(0.21)&(0.06)&(0.06)&(0.06)&(0.06)\\
Mother some coll.&$-0.19$&-&$-0.32$&$-0.39$&-&-&-&-&$0.04$&-&$-0.00$&$0.04$&0.07&-&0.03&0.05\\
&(0.27)&-&(0.44)&(0.49)&-&-&-&-&(0.19)&-&(0.20)&(0.20)&(0.06)&-&(0.06)&(0.06)\\
Mother coll+&$-0.83$&-&$-1.60$&$-1.73$&-&-&-&-&$-0.22$&-&$-0.27$&$-0.23$&0.06&-&0.01&0.01\\
&(0.35)&-&(0.73)&(0.83)&-&-&-&-&(0.18)&-&(0.19)&(0.19)&(0.08)&-&(0.10)&(0.10)\\
Child's age&$-0.36$&$-0.45$&$-0.59$&$-0.65$&$-0.27^{+}$&$-0.34^{+}$&$-0.47^{+}$&$-0.51^{+}$&$-0.06^{+}$&$-0.06^{+}$&$-0.06$&$-0.06^{+}$&-0.02&-0.02&-0.03&-0.03\\
&(0.08)&(0.10)&(0.18)&(0.21)&(0.08)&(0.11)&(0.18)&(0.20)&(0.03)&(0.03)&(0.03)&(0.03)&(0.01)&(0.01)&(0.01)&(0.01)\\
Num. of children 0-5&$0.35$&$0.44$&$0.51$&$0.64$&$0.41$&$0.46$&$0.64$&$0.80$&$0.10$&$0.13$&$0.10$&$0.10$&0.14&0.16&0.15&0.14\\
&(0.18)&(0.23)&(0.31)&(0.36)&(0.25)&(0.30)&(0.42)&(0.49)&(0.12)&(0.14)&(0.12)&(0.13)&(0.05)&(0.05)&(0.05)&(0.05)\\
Type 2&-&$-0.80$&$-1.24$&-&-&-&-&-&-&$0.12$&$0.08$&-&-&0.23&0.23&-\\
&-&(0.41)&(0.61)&-&-&-&-&-&-&(0.34)&(0.31)&-&-&(0.08)&(0.08)&-\\
Type 3&-&$-1.93$&$-2.80^{+}$&-&-&-&-&-&-&$0.08$&$0.03$&-&-&0.02&0.01&-\\
&-&(0.61)&(1.04)&-&-&-&-&-&-&(0.34)&(0.32)&-&-&(0.12)&(0.13)&-\\
$\mu_{k}$&-&-&-&$-2.98^{+}$&-&-&-&-&-&-&-&$0.13^{+}$&-&-&-&-0.18\\
&-&-&-&(1.20)&-&-&-&-&-&-&-&(0.25)&-&-&-&(0.14)\\
Father some coll.&-&-&-&-&$-0.46$&$-0.63$&$-1.04$&$-1.11$&$-0.00$&$0.05$&$0.00$&$-0.03$&0.12&0.11&0.09&0.11\\
&-&-&-&-&(0.41)&(0.50)&(0.73)&(0.82)&(0.24)&(0.29)&(0.25)&(0.25)&(0.08)&(0.08)&(0.08)&(0.08)\\
Father coll+&-&-&-&-&$-0.17$&$-0.23$&$-0.70$&$-0.84$&$-0.66$&$-0.71$&$-0.68$&$-0.73$&0.33&0.34&0.30&0.31\\
&-&-&-&-&(0.36)&(0.42)&(0.64)&(0.72)&(0.21)&(0.25)&(0.24)&(0.24)&(0.08)&(0.08)&(0.08)&(0.08)\\
Year = 2002&-&-&-&-&-&-&-&-&-&-&-&-&0.14&0.16&0.14&0.15\\
&-&-&-&-&-&-&-&-&-&-&-&-&(0.06)&(0.06)&(0.06)&(0.06)\\
\\
\bottomrule\end{tabular}
            \captionsetup{width=1.7\textwidth}
            %\caption*{Note: Superscripts indicate results of Lagrange Multiplier test of individual parameter restrictions. Rejection for a test of size 5\%, 1\% and 0.1\% is indicated by $^{*}$, $^{**}$, and $^{***}$.}
        \end{center}
    \end{table}        

    \begin{table}\footnotesize\caption{\label{res2a}Joint GMM Estimation - Fully Restricted Case, No Borrowing or Saving}
        \begin{center}
            \begin{tabular}{lcccccccccccccccc}\\\toprule
 & \multicolumn{4}{c}{$\epsilon_{\tau,g}$} & \multicolumn{4}{c}{$\epsilon_{x,H}$} & \multicolumn{4}{c}{$\delta_{1}$} & \multicolumn{4}{c}{$\delta_{2}$} \\
&(1)&(2)&(3)&(4)&(1)&(2)&(3)&(4)&(1)&(2)&(3)&(4)&(1)&(2)&(3)&(4)\\\cmidrule(r){2-5}\cmidrule(r){6-9}\cmidrule(r){10-13}\cmidrule(r){14-17}
&$0.31$&$0.25$&$0.19$&$0.18$&$0.54^{+}$&$0.46$&$0.51$&$0.52$&0.07&0.09&0.10&0.11&0.93&0.93&0.93&0.94\\
&(0.05)&(0.05)&(0.05)&(0.05)&(0.08)&(0.08)&(0.08)&(0.08)&(0.04)&(0.04)&(0.04)&(0.04)&(0.01)&(0.01)&(0.01)&(0.01)\\
&&&&&&&&&&&&&&&&\\
 & \multicolumn{4}{c}{$\tilde{\phi}_{m}$: Mother's Time} & \multicolumn{4}{c}{$\tilde{\phi}_{f}$: Father's Time} & \multicolumn{4}{c}{$\tilde{\phi}_{x}$: Childcare} & \multicolumn{4}{c}{$\phi_{\theta}$: TFP} \\
&(1)&(2)&(3)&(4)&(1)&(2)&(3)&(4)&(1)&(2)&(3)&(4)&(1)&(2)&(3)&(4)\\\cmidrule(r){2-5}\cmidrule(r){6-9}\cmidrule(r){10-13}\cmidrule(r){14-17}
Constant&$5.14$&$6.43$&$8.41$&$13.36$&$3.32^{+}$&$3.59$&$4.16$&$4.28$&$-1.20$&$-1.29$&$-1.26$&$-1.46$&-0.98&-1.24&-1.27&-0.71\\
&(0.70)&(1.06)&(1.99)&(4.05)&(0.76)&(0.93)&(1.28)&(1.44)&(0.30)&(0.43)&(0.40)&(0.58)&(0.44)&(0.44)&(0.40)&(0.30)\\
Single&$0.15^{+}$&$0.10^{+}$&$0.05^{+}$&$0.11^{+}$&-&-&-&-&$0.52$&$0.54$&$0.59$&$0.62$&-0.12&-0.11&-0.12&-0.10\\
&(0.24)&(0.29)&(0.38)&(0.41)&-&-&-&-&(0.19)&(0.23)&(0.20)&(0.20)&(0.06)&(0.06)&(0.06)&(0.06)\\
Mother some coll.&$-0.24$&-&$-0.43$&$-0.47$&-&-&-&-&$0.03$&-&$0.00$&$0.04$&0.04&-&0.01&0.00\\
&(0.28)&-&(0.45)&(0.51)&-&-&-&-&(0.18)&-&(0.19)&(0.19)&(0.06)&-&(0.07)&(0.07)\\
Mother coll+&$-0.84$&-&$-1.64$&$-1.74$&-&-&-&-&$-0.23$&-&$-0.28$&$-0.24$&0.03&-&-0.04&-0.09\\
&(0.36)&-&(0.74)&(0.84)&-&-&-&-&(0.17)&-&(0.18)&(0.18)&(0.08)&-&(0.10)&(0.11)\\
Child's age&$-0.37$&$-0.45$&$-0.59^{+}$&$-0.66^{+}$&$-0.28^{+}$&$-0.35^{+}$&$-0.48^{+}$&$-0.53^{+}$&$-0.05$&$-0.06$&$-0.06$&$-0.06$&-0.01&-0.02&-0.02&-0.03\\
&(0.08)&(0.11)&(0.18)&(0.22)&(0.09)&(0.11)&(0.18)&(0.21)&(0.03)&(0.03)&(0.03)&(0.03)&(0.01)&(0.01)&(0.01)&(0.02)\\
Num. of children 0-5&$0.34$&$0.44$&$0.50$&$0.60$&$0.45$&$0.52$&$0.72$&$0.85$&$0.08^{+}$&$0.12^{+}$&$0.08^{+}$&$0.09^{+}$&0.17&0.16&0.16&0.15\\
&(0.19)&(0.23)&(0.31)&(0.36)&(0.25)&(0.31)&(0.43)&(0.50)&(0.11)&(0.13)&(0.12)&(0.12)&(0.05)&(0.05)&(0.05)&(0.05)\\
Type 2&-&$-0.86$&$-1.27$&-&-&-&-&-&-&$0.16$&$0.11$&-&-&0.17&0.15&-\\
&-&(0.42)&(0.62)&-&-&-&-&-&-&(0.33)&(0.30)&-&-&(0.09)&(0.09)&-\\
Type 3&-&$-2.03$&$-2.83$&-&-&-&-&-&-&$0.11$&$0.07$&-&-&-0.04&-0.08&-\\
&-&(0.64)&(1.05)&-&-&-&-&-&-&(0.33)&(0.30)&-&-&(0.13)&(0.14)&-\\
$\mu_{k}$&-&-&-&$-3.05$&-&-&-&-&-&-&-&$0.11$&-&-&-&-0.32\\
&-&-&-&(1.24)&-&-&-&-&-&-&-&(0.24)&-&-&-&(0.16)\\
Father some coll.&-&-&-&-&$-0.54$&$-0.74$&$-1.19$&$-1.28$&$0.01^{+}$&$0.06$&$0.01$&$-0.01$&0.11&0.08&0.06&0.07\\
&-&-&-&-&(0.42)&(0.51)&(0.75)&(0.86)&(0.23)&(0.27)&(0.24)&(0.24)&(0.08)&(0.08)&(0.08)&(0.08)\\
Father coll+&-&-&-&-&$-0.10$&$-0.17$&$-0.62$&$-0.70$&$-0.67$&$-0.73$&$-0.70$&$-0.74$&0.29&0.30&0.26&0.25\\
&-&-&-&-&(0.36)&(0.42)&(0.63)&(0.71)&(0.20)&(0.24)&(0.23)&(0.23)&(0.08)&(0.08)&(0.08)&(0.08)\\
Year = 2002&-&-&-&-&-&-&-&-&-&-&-&-&0.15&0.17&0.16&0.19\\
&-&-&-&-&-&-&-&-&-&-&-&-&(0.06)&(0.06)&(0.06)&(0.07)\\
\\
\bottomrule\end{tabular}
            \captionsetup{width=1.7\textwidth}
            \caption*{Note: Superscripts indicate results of Lagrange Multiplier test of individual parameter restrictions. Rejection for a test of size 5\%, 1\% and 0.1\% is indicated by $^{*}$, $^{**}$, and $^{***}$.}
        \end{center}
    \end{table}      
    
    
    \begin{table}\footnotesize\caption{\label{res3}Joint GMM Estimation - Unrestricted, No Borrowing or Saving}
        \begin{center}
            \begin{tabular}{lccccccc}\toprule
 & \multicolumn{2}{c}{$\epsilon_{\tau,g}$} & \multicolumn{2}{c}{$\epsilon_{x,H}$} & {$\delta_{1}$} & {$\delta_{2}$} & $2N(Q_{N} - \tilde{Q}_{N})$ \\
 & Rel. Dem. & Prod. & Rel. Dem. & Prod. & - & - & - \\\cmidrule(r){2-3}\cmidrule(r){4-5}\cmidrule(r){6-6}\cmidrule(r){7-7}\cmidrule(r){8-8}
&0.19& - &0.53& - &0.14&0.92&5.97\\
&(0.05)& - &(0.08)& - &(0.05)&(0.02)&(0.20)\\
\\
&&&&&&&\\
 & \multicolumn{2}{c}{$\tilde{\phi}_{m}$: Mother's Time} & \multicolumn{2}{c}{$\tilde{\phi}_{f}$: Father's Time} & \multicolumn{2}{c}{$\tilde{\phi}_{x}$: Childcare} &{$\phi_{\theta}$: TFP} \\
 & Rel. Dem. & Prod. & Rel. Dem. & Prod. & Rel. Dem. & Prod. & -  \\\cmidrule(r){2-3}\cmidrule(r){4-5}\cmidrule(r){6-7}\cmidrule(r){8-8}
Constant&8.44& -&4.10& -&-1.25& -&-1.78\\
&(2.00)&&(1.29)&&(0.39)&&(0.78)\\
Single&0.08&3.90& - & -&0.57& -&0.10\\
&(0.38)&(7.16) & &&(0.20)&&(0.22)\\
Type 2&-1.28& -& - & -&0.10& -&0.10\\
&(0.62)& & &&(0.29)&&(0.14)\\
Type 3&-2.78& -& - & -&0.05& -&-0.18\\
&(1.03)& & &&(0.29)&&(0.21)\\
Mother some coll.&-0.40& -& - & -&-0.00& -&-0.02\\
&(0.45)& & &&(0.18)&&(0.08)\\
Mother coll+&-1.62& -& - & -&-0.27& -&-0.06\\
&(0.73)& & &&(0.18)&&(0.14)\\
Child's age&-0.59&-0.64&-0.47&-0.68&-0.06& -&-0.01\\
&(0.18)&(0.43)&(0.18)&(1.86)&(0.03)&&(0.06)\\
Num. of children 0-5&0.46& -&0.68& -&0.09&20.27&0.21\\
&(0.30)&&(0.43)&&(0.12)&(89956857.78)&(0.08)\\
Father some coll.& - & -&-1.18& -&-0.02& -&0.07\\
 & &&(0.75)&&(0.23)&&(0.09)\\
Father coll+& - & -&-0.62& -&-0.69& -&0.18\\
 & &&(0.64)&&(0.22)&&(0.14)\\
Year = 2002& - & -& - & -& - & -&0.19\\
 & & & & & &&(0.07)\\
\\
\bottomrule\end{tabular}
            \captionsetup{width=\textwidth}
            \caption*{Note: the distance metric, $2N(Q_{N} - \tilde{Q}_{N})$, is the difference between the optimally weighted gmm criterion at the restricted estimates and its value at the relaxed estimates. It has a $\chi^{2}$ distribution with degrees of freedom equal to the number of constraints that are relaxed.  Standard errors are indicated in parentheses \emph{except} for the distance metric, which reports a p-value.}
        \end{center}
    \end{table}        


    \begin{table}\footnotesize\caption{\label{res3a}Joint GMM Estimation - Unrestricted, No Binding Constraints}
        \begin{center}
            \begin{tabular}{lccccccc}\toprule
 & \multicolumn{2}{c}{$\epsilon_{\tau,g}$} & \multicolumn{2}{c}{$\epsilon_{x,H}$} & {$\delta_{1}$} & {$\delta_{2}$} & $2N(Q_{N} - \tilde{Q}_{N})$ \\
 & Rel. Dem. & Prod. & Rel. Dem. & Prod. & - & - & - \\\cmidrule(r){2-3}\cmidrule(r){4-5}\cmidrule(r){6-6}\cmidrule(r){7-7}\cmidrule(r){8-8}
&0.21&0.55&0.50& - &0.14&0.83&1.98\\
&(0.05)&(27.77)&(0.09)& - &(0.05)&(0.03)&(0.37)\\
\\
&&&&&&&\\
 & \multicolumn{2}{c}{$\tilde{\phi}_{m}$: Mother's Time} & \multicolumn{2}{c}{$\tilde{\phi}_{f}$: Father's Time} & \multicolumn{2}{c}{$\tilde{\phi}_{x}$: Childcare} &{$\phi_{\theta}$: TFP} \\
 & Rel. Dem. & Prod. & Rel. Dem. & Prod. & Rel. Dem. & Prod. & -  \\\cmidrule(r){2-3}\cmidrule(r){4-5}\cmidrule(r){6-7}\cmidrule(r){8-8}
Constant&7.86& -&4.06& -&-1.19& -&1.36\\
&(1.70)&&(1.17)&&(0.41)&&(0.59)\\
Single&0.08& -& - & -&0.60& -&-0.13\\
&(0.34)& & &&(0.21)&&(0.14)\\
Type 2&-0.87& -& - & -&0.07& -&0.01\\
&(0.50)& & &&(0.30)&&(0.14)\\
Type 3&-2.17& -& - & -&0.10& -&-0.40\\
&(0.81)& & &&(0.31)&&(0.28)\\
Mother some coll.&-0.47& -& - & -&0.01& -&0.07\\
&(0.42)& & &&(0.20)&&(0.11)\\
Mother coll+&-1.55& -& - & -&-0.22& -&-0.11\\
&(0.65)& & &&(0.19)&&(0.21)\\
Child's age&-0.56& -&-0.44& -&-0.07& -&-0.16\\
&(0.16)&&(0.16)&&(0.03)&&(0.04)\\
Num. of children 0-5&0.45& -&0.61& -&0.06& -&0.08\\
&(0.27)&&(0.38)&&(0.12)&&(0.05)\\
Father some coll.& - & -&-0.96& -&-0.00&-18.19&-0.05\\
 & &&(0.66)&&(0.25)&(1094529195.59)&(0.57)\\
Father coll+& - & -&-0.84& -&-0.64& -&0.17\\
 & &&(0.61)&&(0.24)&&(0.10)\\
Year = 2002& - & -& - & -& - & -&-0.33\\
 & & & & & &&(0.06)\\
\\
\bottomrule\end{tabular}
            \captionsetup{width=\textwidth}
            \caption*{Note: the distance metric, $2N(Q_{N} - \tilde{Q}_{N})$, is the difference between the optimally weighted gmm criterion at the restricted estimates and its value at the relaxed estimates. It has a $\chi^{2}$ distribution with degrees of freedom equal to the number of constraints that are relaxed.  Standard errors are indicated in parentheses \emph{except} for the distance metric, which reports a p-value.}
        \end{center}
    \end{table} 

    \begin{table}\footnotesize\caption{\label{res2b}Joint GMM Estimation - Fully Restricted Case, Unconstrained, Older Children}
        \begin{center}
            \begin{tabular}{lcccccccccccc}\\\toprule
 & \multicolumn{4}{c}{$\rho$} & \multicolumn{4}{c}{$\delta_{1}$} & \multicolumn{4}{c}{$\delta_{2}$} \\
&(1)&(2)&(3)&(4)&(1)&(2)&(3)&(4)&(1)&(2)&(3)&(4)\\\cmidrule(r){2-5}\cmidrule(r){6-9}\cmidrule(r){10-13}
&$0.24$&$0.18$&$0.11$&$0.06$&0.02&0.01&0.02&0.01&0.92&0.93&0.92&0.93\\
&(0.06)&(0.06)&(0.07)&(0.07)&(0.03)&(0.03)&(0.03)&(0.03)&(0.01)&(0.01)&(0.01)&(0.01)\\
&&&&&&&&&&&&\\
 & \multicolumn{4}{c}{$\tilde{\phi}_{m}$: Mother's Time} & \multicolumn{4}{c}{$\tilde{\phi}_{f}$: Father's Time} & \multicolumn{4}{c}{$\tilde{\phi}_{\theta}$: TFP} \\
&(1)&(2)&(3)&(4)&(1)&(2)&(3)&(4)&(1)&(2)&(3)&(4)\\\cmidrule(r){2-5}\cmidrule(r){6-9}\cmidrule(r){10-13}
Constant&$6.06$&$7.51$&$12.63$&$36.51$&$3.66$&$4.45$&$6.53$&$11.26$&2.18&2.40&2.25&2.47\\
&(2.18)&(3.08)&(7.84)&(40.30)&(2.39)&(3.21)&(5.91)&(13.97)&(0.35)&(0.35)&(0.36)&(0.35)\\
Single&$0.43$&$0.49$&$0.52$&$0.83$&-&-&-&-&-0.08&-0.07&-0.07&-0.07\\
&(0.46)&(0.61)&(1.00)&(1.93)&-&-&-&-&(0.06)&(0.06)&(0.06)&(0.06)\\
Mother some coll.&$-0.76$&-&$-1.71$&$-2.84$&-&-&-&-&-0.04&-&-0.02&-0.01\\
&(0.60)&-&(1.63)&(3.94)&-&-&-&-&(0.06)&-&(0.06)&(0.06)\\
Mother coll+&$-1.23$&-&$-3.21$&$-5.63$&-&-&-&-&-0.07&-&-0.05&-0.04\\
&(0.73)&-&(2.58)&(7.07)&-&-&-&-&(0.07)&-&(0.07)&(0.07)\\
Child's age&$-0.49$&$-0.59$&$-0.96$&$-1.76$&$-0.37$&$-0.53$&$-0.94$&$-1.82$&-0.12&-0.12&-0.12&-0.12\\
&(0.21)&(0.30)&(0.69)&(2.04)&(0.24)&(0.35)&(0.77)&(2.24)&(0.03)&(0.03)&(0.03)&(0.03)\\
Num. of children 0-5&$0.37$&$0.61$&$0.85$&$1.96$&$0.62$&$0.79$&$1.33$&$2.21$&-0.00&-0.02&-0.02&-0.01\\
&(0.44)&(0.59)&(1.01)&(2.63)&(0.54)&(0.71)&(1.35)&(3.07)&(0.06)&(0.06)&(0.06)&(0.06)\\
Type 2&-&$-1.67$&$-2.99$&-&-&-&-&-&-&-0.06&-0.07&-\\
&-&(1.05)&(2.37)&-&-&-&-&-&-&(0.07)&(0.07)&-\\
Type 3&-&$-3.20$&$-5.48$&-&-&-&-&-&-&-0.07&-0.12&-\\
&-&(1.63)&(3.99)&-&-&-&-&-&-&(0.13)&(0.11)&-\\
$\mu_{k}$&-&-&-&$-11.02$&-&-&-&-&-&-&-&-0.08\\
&-&-&-&(13.25)&-&-&-&-&-&-&-&(0.12)\\
Father some coll.&-&-&-&-&$-1.11$&$-1.72$&$-3.23$&$-6.53$&-0.03&-0.05&-0.04&-0.02\\
&-&-&-&-&(0.84)&(1.19)&(2.78)&(8.33)&(0.08)&(0.08)&(0.08)&(0.08)\\
Father coll+&-&-&-&-&$-1.08$&$-1.76$&$-3.50$&$-7.27$&0.14&0.12&0.14&0.14\\
&-&-&-&-&(0.77)&(1.12)&(2.92)&(9.12)&(0.08)&(0.08)&(0.07)&(0.08)\\
Year = 2002&-&-&-&-&-&-&-&-&-0.31&-0.32&-0.30&-0.32\\
&-&-&-&-&-&-&-&-&(0.04)&(0.04)&(0.04)&(0.04)\\
\\
\bottomrule\end{tabular}
            \captionsetup{width=1.7\textwidth}
            %\caption*{Note: Superscripts indicate results of Lagrange Multiplier test of individual parameter restrictions. Rejection for a test of size 5\%, 1\% and 0.1\% is indicated by $^{*}$, $^{**}$, and $^{***}$.}
        \end{center}
    \end{table}      


    \begin{table}\footnotesize\caption{\label{res4}Joint GMM Estimation - Mother's Share Unrestricted, No Borrowing or Saving}
        \begin{center}
            \begin{tabular}{lccccccc}\toprule
 & \multicolumn{2}{c}{$\epsilon_{\tau,g}$} & \multicolumn{2}{c}{$\epsilon_{x,H}$} & {$\delta_{1}$} & {$\delta_{2}$} & $2N(Q_{N} - \tilde{Q}_{N})$ \\
 & Rel. Dem. & Prod. & Rel. Dem. & Prod. & - & - & - \\\cmidrule(r){2-3}\cmidrule(r){4-5}\cmidrule(r){6-6}\cmidrule(r){7-7}\cmidrule(r){8-8}
&0.21& - &0.50& - &0.13&0.87&0.04\\
&(0.05)& - &(0.09)& - &(0.04)&(0.02)&(0.85)\\
\\
&&&&&&&\\
 & \multicolumn{2}{c}{$\tilde{\phi}_{m}$: Mother's Time} & \multicolumn{2}{c}{$\tilde{\phi}_{f}$: Father's Time} & \multicolumn{2}{c}{$\tilde{\phi}_{x}$: Childcare} &{$\phi_{\theta}$: TFP} \\
 & Rel. Dem. & Prod. & Rel. Dem. & Prod. & Rel. Dem. & Prod. & -  \\\cmidrule(r){2-3}\cmidrule(r){4-5}\cmidrule(r){6-7}\cmidrule(r){8-8}
Constant&7.91&8.41&4.35& -&-1.18& -&2.24\\
&(1.71)&(10.34)&(1.21)&&(0.41)&&(1.18)\\
Single&0.18& -& - & -&0.63& -&-0.20\\
&(0.34)& & &&(0.21)&&(0.25)\\
Type 2&-1.09& -& - & -&-0.02& -&0.14\\
&(0.54)& & &&(0.30)&&(0.31)\\
Type 3&-2.26& -& - & -&-0.02& -&-0.15\\
&(0.83)& & &&(0.31)&&(0.59)\\
Mother some coll.&-0.51& -& - & -&0.08& -&0.07\\
&(0.42)& & &&(0.20)&&(0.18)\\
Mother coll+&-1.60& -& - & -&-0.12& -&-0.05\\
&(0.67)& & &&(0.20)&&(0.43)\\
Child's age&-0.55& -&-0.48& -&-0.07& -&-0.18\\
&(0.15)&&(0.16)&&(0.03)&&(0.13)\\
Num. of children 0-5&0.42& -&0.55& -&0.05& -&0.06\\
&(0.27)&&(0.38)&&(0.12)&&(0.06)\\
Father some coll.& - & -&-1.16& -&-0.04& -&-0.03\\
 & &&(0.69)&&(0.25)&&(0.09)\\
Father coll+& - & -&-0.88& -&-0.58& -&0.17\\
 & &&(0.62)&&(0.23)&&(0.11)\\
Year = 2002& - & -& - & -& - & -&-0.36\\
 & & & & & &&(0.06)\\
\\
\bottomrule\end{tabular}
            \captionsetup{width=\textwidth}
            \caption*{Note: the distance metric, $2N(Q_{N} - \tilde{Q}_{N})$, is the difference between the optimally weighted gmm criterion at the restricted estimates and its value at the relaxed estimates. It has a $\chi^{2}$ distribution with degrees of freedom equal to the number of constraints that are relaxed.  Standard errors are indicated in parentheses \emph{except} for the distance metric, which reports a p-value.}
        \end{center}
    \end{table}        


    \begin{table}\footnotesize\caption{\label{res4a}Joint GMM Estimation - Mother's Share Unrestricted, No Binding Constraints}
        \begin{center}
            \begin{tabular}{lccccccc}\toprule
 & \multicolumn{2}{c}{$\epsilon_{\tau,g}$} & \multicolumn{2}{c}{$\epsilon_{x,H}$} & {$\delta_{1}$} & {$\delta_{2}$} & $2N(Q_{N} - \tilde{Q}_{N})$ \\
 & Rel. Dem. & Prod. & Rel. Dem. & Prod. & - & - & - \\\cmidrule(r){2-3}\cmidrule(r){4-5}\cmidrule(r){6-6}\cmidrule(r){7-7}\cmidrule(r){8-8}
&0.20& - &0.50& - &0.14&0.83&0.02\\
&(0.05)& - &(0.09)& - &(0.04)&(0.02)&(0.88)\\
\\
&&&&&&&\\
 & \multicolumn{2}{c}{$\tilde{\phi}_{m}$: Mother's Time} & \multicolumn{2}{c}{$\tilde{\phi}_{f}$: Father's Time} & \multicolumn{2}{c}{$\tilde{\phi}_{x}$: Childcare} &{$\phi_{\theta}$: TFP} \\
 & Rel. Dem. & Prod. & Rel. Dem. & Prod. & Rel. Dem. & Prod. & -  \\\cmidrule(r){2-3}\cmidrule(r){4-5}\cmidrule(r){6-7}\cmidrule(r){8-8}
Constant&8.43&8.80&4.28& -&-1.21& -&1.28\\
&(1.99)&(11.14)&(1.30)&&(0.41)&&(0.87)\\
Single&0.08& -& - & -&0.61& -&-0.14\\
&(0.37)& & &&(0.21)&&(0.27)\\
Type 2&-0.98& -& - & -&0.08& -&0.04\\
&(0.56)& & &&(0.30)&&(0.24)\\
Type 3&-2.44& -& - & -&0.10& -&-0.29\\
&(0.95)& & &&(0.31)&&(0.48)\\
Mother some coll.&-0.53& -& - & -&0.00& -&0.10\\
&(0.46)& & &&(0.20)&&(0.16)\\
Mother coll+&-1.77& -& - & -&-0.22& -&-0.02\\
&(0.76)& & &&(0.19)&&(0.37)\\
Child's age&-0.62& -&-0.49& -&-0.07& -&-0.13\\
&(0.18)&&(0.18)&&(0.03)&&(0.11)\\
Num. of children 0-5&0.50& -&0.67& -&0.07& -&0.08\\
&(0.30)&&(0.43)&&(0.12)&&(0.06)\\
Father some coll.& - & -&-1.10& -&-0.01& -&0.01\\
 & &&(0.74)&&(0.25)&&(0.09)\\
Father coll+& - & -&-1.01& -&-0.64& -&0.21\\
 & &&(0.69)&&(0.24)&&(0.11)\\
Year = 2002& - & -& - & -& - & -&-0.34\\
 & & & & & &&(0.06)\\
\\
\bottomrule\end{tabular}
            \captionsetup{width=\textwidth}
            \caption*{Note: the distance metric, $2N(Q_{N} - \tilde{Q}_{N})$, is the difference between the optimally weighted gmm criterion at the restricted estimates and its value at the relaxed estimates. It has a $\chi^{2}$ distribution with degrees of freedom equal to the number of constraints that are relaxed.  Standard errors are indicated in parentheses \emph{except} for the distance metric, which reports a p-value.}
        \end{center}
    \end{table}        

    \begin{table}\footnotesize\caption{Direct GMM Estimation - Relaxed Demand Specification}
        \begin{center}
            \begin{tabular}{lcccc}\\\toprule
 & \multicolumn{1}{c}{$\rho$} & \multicolumn{1}{c}{$\gamma $} & \multicolumn{1}{c}{$\delta_{1}$} & \multicolumn{1}{c}{$\delta_{2}$} \\
&(1)&(1)&(1)&(1)\\\cmidrule(r){2-2}\cmidrule(r){3-3}\cmidrule(r){4-4}\cmidrule(r){5-5}
&[-3.00, -0.04]&[-6.79, 2.17]&[-0.13, 0.07]&[0.71, 0.96]\\
&&&&\\
 & \multicolumn{1}{c}{$\tilde{\phi}_{m}$: Mother's Time} & \multicolumn{1}{c}{$\tilde{\phi}_{f}$: Father's Time} & \multicolumn{1}{c}{$\tilde{\phi}_{x}$: Childcare} & \multicolumn{1}{c}{$\phi_{\theta}$: TFP} \\
&(1)&(1)&(1)&(1)\\\cmidrule(r){2-2}\cmidrule(r){3-3}\cmidrule(r){4-4}\cmidrule(r){5-5}
Constant&[-3.88, 2.75]&[-3.93, 0.99]&[-2.59, 1.80]&[-4.65, 0.58]\\
Single&-&-&[-4.40, 2.43]&[-0.01, 1.84]\\
Mother some coll.&-&-&-&[-3.19, 0.89]\\
Mother coll+&-&-&-&[-0.16, 2.17]\\
Father some coll.&-&-&-&[-1.16, 2.97]\\
Father coll+&-&-&-&[-0.11, 2.35]\\
Child's age&-&-&-&[-0.01, 0.52]\\
Num. of children 0-5&-&-&-&[0.45, 4.13]\\
Year = 2002&-&-&-&[-0.71, -0.25]\\
\\
\bottomrule\end{tabular}
            \captionsetup{width=\textwidth}
            \caption*{This table reports the 10th and 90th percentiles of the bootstrapped distribution of parameter estimates from the direct estimation method. To avoid convergence issues, each bootstrap trial is terminated after 100 LBFGS iterations followed by 10 iterations of Newton's method.}
        \end{center}
    \end{table}        


\end{landscape}

\begin{table}\footnotesize\caption{Monte Carlo Simulation - Direct vs Indirect Methods}
    \begin{center}
        \begin{tabular}{lcccccccc} \\\toprule
 & \multicolumn{8}{c}{Results for $\rho$} \\ 
 & \multicolumn{4}{c}{Bias} & \multicolumn{4}{c}{Std. Dev.} \\ 
& (1)& (2)& (3)& (4)& (1)& (2)& (3)& (4)\\ 
\cmidrule(r){2-5}\cmidrule(r){6-9}$N$ = 500 & 17.08 & 23.76 & 0.23 & 13.96 & 22.49 & 23.69 & 1.19 & 20.67\\ 
$N$ = 1000 & 13.73 & 27.45 & 0.09 & 9.96 & 20.85 & 23.52 & 0.70 & 18.03\\ 
$N$ = 5000 & 2.01 & 26.23 & -0.00 & 0.75 & 8.05 & 22.05 & 0.29 & 4.54\\ 
&&&&&&&& \\ 
 & \multicolumn{8}{c}{Results for $a$} \\ 
 & \multicolumn{4}{c}{Bias} & \multicolumn{4}{c}{Std. Dev.} \\ 
& (1)& (2)& (3)& (4)& (1)& (2)& (3)& (4)\\ 
\cmidrule(r){2-5}\cmidrule(r){6-9}$N$ = 500 & -0.00 & -0.01 & -0.00 & -0.01 & 0.35 & 0.41 & 0.04 & 0.31\\ 
$N$ = 1000 & 0.00 & -0.01 & 0.00 & 0.02 & 0.32 & 0.38 & 0.03 & 0.26\\ 
$N$ = 5000 & 0.01 & 0.00 & -0.00 & -0.00 & 0.15 & 0.29 & 0.01 & 0.10\\ 
&&&&&&&& \\ 
 & \multicolumn{8}{c}{Results for $\delta$} \\ 
 & \multicolumn{4}{c}{Bias} & \multicolumn{4}{c}{Std. Dev.} \\ 
& (1)& (2)& (3)& (4)& (1)& (2)& (3)& (4)\\ 
\cmidrule(r){2-5}\cmidrule(r){6-9}$N$ = 500 & -0.00 & -0.00 & -0.00 & -0.00 & 0.02 & 0.02 & 0.02 & 0.02\\ 
$N$ = 1000 & -0.00 & -0.00 & 0.00 & 0.00 & 0.01 & 0.01 & 0.01 & 0.01\\ 
$N$ = 5000 & -0.00 & -0.00 & -0.00 & -0.00 & 0.00 & 0.00 & 0.00 & 0.01\\ 
\bottomrule\end{tabular}
        \captionsetup{width=\textwidth}
        \caption*{This table presents results from 500 Monte Carlo samples of each estimator. Estimator 1 assumes inputs are perfectly measured and performs NLLS on production outcomes. Estimator 2 assumes measurement error in input 2 and uses relative demand observations to impute the second input, but does not impose cross-equation restrictions between demand and production when performing NLLS. Estimator 3 imposes the cross-equation restrictions. Estimator 4 implements estimator 1 assuming the true value of $\rho$.}
    \end{center}
\end{table}

\begin{table}\footnotesize\caption{Monte Carlo Simulation - Increasing Relative Price Variation}
    \begin{center}
        \begin{tabular}{lcccccc} \\\toprule
 & \multicolumn{2}{c}{$\rho$} & \multicolumn{2}{c}{$a$} & \multicolumn{2}{c}{$\delta$} \\ 
 & Bias & Std. Dev. & Bias & Std. Dev. & Bias & Std. Dev. \\
\cmidrule(r){2-3}\cmidrule(r){4-5}\cmidrule(r){6-7} \\
Method 1, $N=500$ & 23.62 & 23.68 & -0.00 & 0.41 & -0.01 & 0.03\\
Method 1, $N=1000$ & 22.35 & 23.64 & -0.01 & 0.38 & -0.00 & 0.02\\
Method 1, $N=2000$ & 18.97 & 23.10 & 0.01 & 0.36 & 0.00 & 0.02\\
Method 2, $N=500$ & 24.53 & 24.03 & 0.02 & 0.42 & -0.00 & 0.03\\
Method 2, $N=1000$ & 24.73 & 24.04 & -0.01 & 0.40 & -0.00 & 0.02\\
Method 2, $N=2000$ & 23.04 & 23.83 & 0.01 & 0.37 & -0.00 & 0.02\\
Method 3, $N=500$ & 0.08 & 0.52 & -0.00 & 0.04 & -0.00 & 0.03\\
Method 3, $N=1000$ & 0.03 & 0.35 & -0.00 & 0.03 & 0.00 & 0.02\\
Method 3, $N=2000$ & 0.03 & 0.25 & 0.00 & 0.02 & 0.00 & 0.02\\
Method 4, $N=500$ & 0.00 & 0.00 & -0.01 & 0.37 & -0.00 & 0.03\\
Method 4, $N=1000$ & 0.00 & 0.00 & -0.01 & 0.32 & -0.00 & 0.02\\
Method 4, $N=2000$ & 0.00 & 0.00 & 0.02 & 0.27 & 0.00 & 0.02\\
\bottomrule\end{tabular}
        \captionsetup{width=\textwidth}
        \caption*{Same simulation results with $\sigma_{\pi}$ doubled.}
    \end{center}
\end{table}

\begin{table}\footnotesize\caption{Monte Carlo Simulation - Increasing Idiosyncratic Preference Variation}
    \begin{center}
        \begin{tabular}{lcccccc} \\\toprule
 & \multicolumn{2}{c}{$\rho$} & \multicolumn{2}{c}{$a$} & \multicolumn{2}{c}{$\delta$} \\ 
 & Bias & Std. Dev. & Bias & Std. Dev. & Bias & Std. Dev. \\
\cmidrule(r){2-3}\cmidrule(r){4-5}\cmidrule(r){6-7} \\
Method 1, $N=500$ & 23.62 & 23.68 & -0.00 & 0.41 & -0.01 & 0.03\\
Method 1, $N=1000$ & 22.35 & 23.64 & -0.01 & 0.38 & -0.00 & 0.02\\
Method 1, $N=2000$ & 18.97 & 23.10 & 0.01 & 0.36 & 0.00 & 0.02\\
Method 2, $N=500$ & 24.53 & 24.03 & 0.02 & 0.42 & -0.00 & 0.03\\
Method 2, $N=1000$ & 24.73 & 24.04 & -0.01 & 0.40 & -0.00 & 0.02\\
Method 2, $N=2000$ & 23.04 & 23.83 & 0.01 & 0.37 & -0.00 & 0.02\\
Method 3, $N=500$ & 0.08 & 0.52 & -0.00 & 0.04 & -0.00 & 0.03\\
Method 3, $N=1000$ & 0.03 & 0.35 & -0.00 & 0.03 & 0.00 & 0.02\\
Method 3, $N=2000$ & 0.03 & 0.25 & 0.00 & 0.02 & 0.00 & 0.02\\
Method 4, $N=500$ & 0.00 & 0.00 & -0.01 & 0.37 & -0.00 & 0.03\\
Method 4, $N=1000$ & 0.00 & 0.00 & -0.01 & 0.32 & -0.00 & 0.02\\
Method 4, $N=2000$ & 0.00 & 0.00 & 0.02 & 0.27 & 0.00 & 0.02\\
\bottomrule\end{tabular}
        \captionsetup{width=\textwidth}
        \caption*{Same simulation results with all residual variation in relative demand attributed to true variation instead of measurement error.}
    \end{center}
\end{table}

\end{document}





%%% Local Variables:
%%% mode: latex
%%% TeX-master: t
%%% End:

\documentclass{article}
\usepackage{booktabs,caption}
\usepackage{geometry}
\usepackage{color}
\usepackage{pdflscape}

\title{Summary of GMM Estimates}
\author{your old pal Jo}

\begin{document}

\maketitle
\section*{Preliminaries}
\begin{itemize}
    \item Let there be $N$ children indexed by $n$.
    \item Let the vector of production parameters to be estimated be $(\rho,\gamma,\phi,\phi_{\theta},\delta)$ where $(\rho,\gamma)$ are the elasticity parameters, $\phi$ represents the coefficients on observables that determine the factor shares, $\phi_{\theta}$ is the vector of coefficients that determine $\theta$ (TFP), and $\delta$ is the vector of cobb-douglas factor shares in the outer aggregator.
    \item Here we assume:
    \[ f(\cdot) = \left[\left(a_{m}\tau_{m}^\rho+a_{f}\tau_{f}^\rho+a_{g}g^\rho\right)^{\gamma / \rho}(1-a_{Y}) + a_{Y}Y_{c}^{\gamma}\right]^{1/\gamma} \]
    where
    \begin{eqnarray}
        a_{x} = \frac{\exp(\phi_{x}Z)}{1+\exp(\phi_{m}Z)+\exp(\phi_{f}Z)},\ \forall x\in\{m,f\} \nonumber \\
        a_{g} = \frac{1}{1+\exp(\phi_{m}Z)+\exp(\phi_{f}Z)} \nonumber \\
        a_{Y} = \frac{\exp(\phi_{Y}Z)}{1+\exp(\phi_{Y}Z)} \nonumber
    \end{eqnarray}
    This normalization of factor shares differs slightly from the text, but is necessary to guarantee each aggregator has a Cobb-Douglas form as $\rho$ and $\gamma$ approach 0.
    \item We use $\tilde{\cdot}$ to indicate the \emph{perceived} value of a production parameter as is implied by the population demand system.
    \item We let $\Phi_{x,y,n,t}$ be the generic notation for the ratio between any two inputs $x$ and $y$ at time $t$, which depends on the input prices $\Pi_{t}$, factor shares $(\tilde{a}_{g,n,t},\tilde{a}_{m,n,t},\tilde{a}_{f,n,t},\tilde{a}_{Y,n,t})$ and elasticity parametes $(\tilde{\rho},\tilde{\gamma})$. The subscript $n$ and $t$ denotes implicit dependence of the ratio on prices and parameters, though when necessary we make explicit this dependence below.
    \item The original draft derives expressions for many objects with goods inputs $g_{n,t}$ at the denominator, and in this case the notation $\Phi_{x,n,t}$ is used where it is assumed that this gives the ratio of input $x$ to goods.
\end{itemize}

\section*{Relative Demand Moments}
As before, relative demand is summarized by the parameters $\tilde{\omega}$ which can be estimated by forming moment conditions on the difference between observed input ratios and those predicted by prices (with deviations are explained only by measurement error in inputs).

Formally, moment conditions for relative demand are given as in Equation (36) in the first draft, with the specific residuals for each year given by Table (14) in the paper. Each residual is interacted with the full set of covariates that are allowed to affect factor shares, as well as the relevant relative prices for that particular residual. {\color{red}No instruments for prices are used yet for these estimates.}

Let $g_{1,N}(\tilde{\omega})$ indicate the vector of moments from these relative demand residuals each interacted with their respective instruments. The estimates in Table \ref{res1} below report the solution to:

\[ \hat{\tilde{\omega}} = \arg\min\frac{1}{2}g_{1,N}(\tilde{\omega})^{T}\hat{\Omega}^{-1}g_{1,N}(\tilde{\omega}) \]
where $\hat{\Omega}$ is an estimate of the variance of the moment conditions from a first stage estimate using the identity weighting matrix.

Some of the specifications below rely on grouped fixed effect estimates from the mother's wage equation, assuming the model:
\[ \log(W_{mt}) = \mu_{k(m)} + X_{mt}\beta + \epsilon_{mt} \]
where $X_{mt}$ includes education dummies, age and age squared, and $k(m)\in\{1,2,...,K\}$ is mother $m$'s type. It is estimated via an iterative clustering algorithm. Specifications (2) and (3) use dummy variables indicating each mother's type from a clustering routine with $K=3$. Specifications (4) and (5) instead uses the estimated value of $\mu_{k(m)}$ for each $m$, with $K=10$, which enters parametrically.

\subsection*{Testing correlation in residuals}
Assuming that measurement error is not correlated over time, any correlation in the relative demand residuals over time can be explained by persistent unobserved heterogeneity in demand. The results below report a simple test of the null hypothesis that no such heterogeneity exists, which implies that the test statistic:
\[ T_{N} = \sqrt{N}\frac{\sum_{n}\xi_{Y,m,n,0}\xi_{Y,m,n,5}}{\sqrt{s^2_{Y,m,0}s^2_{Y,m,5}}} \]
is asymptotically standard normal under the null. Here, $\xi_{Y,m,n,t}$ is the residual in the demand for childcare relative to mother's time for child $n$ at time $t$ and $s^2_{Y,m,t}$ is the corresponding sample variance across individuals.

\section*{Joint Estimation with Relative Demand and Achievement Moments}

Next, the intratemporal moment conditions are expanded to include intertemporal moment conditions on skill outcomes. Differently from the original approach, the overall input $X_{n,t}$ may be imputed using separate relative demand parameters. For the case of single mothers (for example) this is:

\[ X_{n,t} = \left(\left(a_{m,n,t} + a_{g,n,t}\Phi_{g,m,n,t}^\rho\right)^{\gamma/
\rho}(1-a_{Y,n,t}) + a_{Y,n,t}\Phi_{Y,m,n,t}^\gamma\right)^{1/\gamma}\tau_{m,n,t}
 \]
where the $\Phi$ terms depend on the perceived production parameters $\tilde{\omega}$ and total investment also depends on their true values, $\omega$. Write:
\[ X_{n,t} = \Phi_{X,m,n,t}\tau_{m,n,t}.\]
such that dependence on prices, observables, and parameters is made implicit.
As in the previous draft, we must iterate the outcome equation forward and using a modeling assumption to impute investment in missing periods. Let $\overline{p}(\Pi_{n,t},Z_{n,t},\omega,\tilde{\omega})$ indicate the \emph{effective} composite price of $X$, which can be defined as dollars of expenditure per investment unit:
\[ \overline{p}(\Pi_{n,t},Z_{n,t},\omega,\tilde{\omega}) = \frac{p_{n,t}g_{n,t}+P_{Y,n,t}Y_{c,n,t}+W_{m,n,t}\tau_{m,n,t}+W_{f,n,t}\tau_{f,n,t}}{X_{n,t}} \]
which due to the proportional investment rules, takes the same form as equation (9) in the first draft:
\[ \overline{p}(\Pi_{n,t},Z_{n,t},\omega,\tilde{\omega}) = \frac{p_{n,t}+P_{Y,n,t}\Phi_{Y,n,t}+W_{m,n,t}\Phi_{m,n,t}+W_{f,n,t}\Phi_{f,n,t}}{\left[\left(a_{m,n,t}\Phi_{m,n,t}^\rho+a_{f,n,t}\Phi_{f,n,t}^\rho+a_{g,n,t}\right)^{\gamma / \rho}(1-a_{Y,n,t}) + a_{Y,n,t}\Phi_{Y,n,t}^{\gamma}\right]^{1/\gamma}} \]
where one key difference is that the investment ratios: $(\Phi_{m,n,t},\Phi_{f,n,t},\Phi_{Y,n,t})$ are calculated using the perceived parameters $\tilde{\omega}$ and not their real values $\omega$.

\subsection*{No Binding Borrowing Constraints}

To impute investment when borrowing constraints do not bind, we begin with the implication that total expenditures in any two years are proportional to each other:

\[ E_{n,t} \propto E_{n,0} \]
where we can use the derived price index to write this as:
\[ \overline{p}(\Pi_{n,t},Z_{n,t},\omega,\tilde{\omega})X_{n,t} \propto \overline{p}(\Pi_{n,0},Z_{n,0},\omega,\tilde{\omega})X_{n,0} \]
Rearranging and using $\Phi_{X,m}$ then gives:
\[ X_{n,t} \propto \frac{\overline{p}(\Pi_{n,0},Z_{n,0},\omega,\tilde{\omega})\Phi_{X,m,n,0}\tau_{m,n,0}}{\overline{p}(\Pi_{n,t},Z_{n,t},\omega,\tilde{\omega})}\]
which results in an estimating equation:
\[\lambda^{-1}_{S}S_{n,5} = Z_{n,0}\tilde{\phi}_\theta + \sum_{t=0}^4\delta_{2}^{4-t}\delta_{1}\ln\left(\frac{\overline{p}(\Pi_{n,0},Z_{n,0},\omega,\tilde{\omega})\Phi_{X,m,n,0}\tau^{o}_{m,n,0}}{\overline{p}(\Pi_{n,t},Z_{n,t},\omega,\tilde{\omega})}\right) + \delta^5_{2}\lambda_{S}^{-1}S_{n,0} + \xi_{S,n} \]
where $\xi_{S,n}$ collects measurement error in $\tau_{m,n,0}$ as well as measurement error in $S_{n,5}$ and $S_{n,0}$. To account for measurement error in both time investment and skills, future observations of time investment ($\tau_{m,n,5}$) are used to instrument for $\tau_{m,n,0}$, and the second measure of skills (applied problems if letter-word is used as the outcome, and vice versa) is used to instrument for the outcome score.

\subsection*{No Borrowing or Saving}
In the opposite case, we derive in the paper that:
\[ E_{n,t} \propto W_{m,n,t}+W_{f,n,t}+y_{n,t} \]
so that investment can be imputed as:
\[ X_{n,t} \propto \frac{W_{m,n,t}+W_{f,n,t}+y_{n,t}}{\overline{p}(\Pi_{n,t},Z_{n,t},\omega,\tilde{\omega})} \]
which results in an estimating equation:
\[\lambda^{-1}_{S}S_{n,5} = Z_{n,0}\tilde{\phi}_\theta + \sum_{t=1}^4\delta_{2}^{4-t}\delta_{1}\ln\left(\frac{W_{m,n,t}+W_{f,n,t}+y_{n,t}}{\overline{p}(\Pi_{n,t},Z_{n,t},\omega,\tilde{\omega})}\right) + \delta_{1}\delta_{2}^4\ln\left(\Phi_{X,m,n,0}\tau^{o}_{m,n,0}\right) + \delta^5_{2}\lambda_{S}^{-1}S_{n,0} + \xi_{S,n} \]

\subsection*{Estimation and Results}
Let $\Delta=(\delta,\phi_{\theta},\omega,\tilde{\omega})$ be the full set of parameters. The final vector of moments, $g_{2,N}(\Delta)$ consists of the pair of residuals $(\xi_{LW,n},\xi_{AP,n})$ each interacted with:
\begin{itemize}
    \item The vector of observables allowed to determine $\theta$.
    \item Mother's future time investment, $\log(\tau_{m,n,5})$, and log of full income: $\log(W_{m,n,t}+W_{f,n,t})$ as instruments for mother's time investment.
    \item A full set of interactions between relative prices and the observables that are specified to affect factor shares.
\end{itemize}

Let $g_{N}(\Delta) = [g_{1,N}(\tilde{\omega})^\prime,\ g_{2,n}(\Delta)^\prime]^\prime$ be the full vector of moments. Table \ref{res2} presents estimates that are the solution to
\[ \hat{\Delta} = \arg\min_{\Delta}g_{N}(\Delta)'\hat{\Omega}^{-1}g_{N}(\Delta) \]
where the assumption of unconstrained households is used to impute investment and $\hat{\Omega}$ is the estimated covariance of the moments from a first stage estimate with the identity weighting matrix. Similarly, Table \ref{res3} presents results of the same estimation problem where the assumption that households cannot borrow or save is used to impute investment.

We will find that there is very little information to identify $\omega$ separately from $\tilde{\omega}$ and so both Tables \ref{res2} and \ref{res3} impose the constraint that $\omega=\tilde{\omega}$. Accordingly, we use a Lagrange Multiplier (LM) statistic\footnote{See Newey \& McFadden (1994), Section 9.2} to test equality restrictions on individual components of $\omega$ and $\tilde{\omega}$. As is evident in the tables, in most cases we cannot reject the null of equality. 

For our preferred specification (specification 3) we re-estimate the model having relaxed the particular restrictions that are rejected in the first case. Table \ref{res4} presents the results and shows that even for the case of those parameters that fail the LM test, these parameters cannot be identified separately with sufficient precision. Furthermore, a ``distance metric'' test statistic\footnote{Once again, Newey \& McFadden (1994) Section 9.2}, which follows a $\chi^2$ distribution under the null hypothesis, fails to reject the joint null of equality for these parameters. We conclude that there is not sufficient evidence to reject the null of equality $\tilde{\omega}=\omega$.


\newgeometry{top=0in,bottom=0in}
\begin{table}\footnotesize\caption{\label{res1}GMM Estimation of Relative Demand System}
    \begin{center}
        \begin{tabular}{lccccccc}\\\toprule
&(1)&(2)&(3)&(4)&(5)&(6)&\\\cmidrule(r){2-8}$\rho$&-1.56&-1.52&-1.78&-2.34&-2.35&-2.28&\\
&(0.46)&(0.45)&(0.56)&(0.84)&(0.95)&(0.84)&\\
$\gamma$&-1.47&-1.37&-1.93&-1.61&-1.98&-1.68&\\
&(0.56)&(0.56)&(0.78)&(0.69)&(0.83)&(0.75)&\\
& \multicolumn{7}{c}{$\phi_{m}$: Mother's Time}\\\cmidrule(r){2-8}Married&3.73&3.95&5.94&8.77&4.44&5.00&\\&(1.85)&(2.05)&(2.49)&(4.20)&(2.77)&(3.00)&\\Single&3.80&3.89&5.96&8.40&4.41&4.78&\\&(1.50)&(1.61)&(2.10)&(3.52)&(2.24)&(2.36)&\\Mother: Some College&-0.32&-0.34&-&-0.58&-&-0.57&\\&(0.30)&(0.30)&&(0.44)&&(0.44)&\\Mother: College+&0.32&0.10&-&0.10&-&0.04&\\&(0.32)&(0.31)&&(0.42)&&(0.42)&\\Child Age&-0.18&-0.18&-0.21&-0.26&-0.26&-0.25&\\&(0.07)&(0.07)&(0.07)&(0.11)&(0.11)&(0.11)&\\Num. Children 0-5&-1.11&-1.11&-1.11&-1.48&-1.42&-1.43&\\&(0.41)&(0.40)&(0.41)&(0.64)&(0.64)&(0.62)&\\Type 2&-&-&-2.15&-2.84&-&-&\\&&&(0.83)&(1.21)&&&\\Type 3&-&-&-2.56&-3.34&-&-&\\&&&(0.90)&(1.32)&&&\\$\mu_{k}$&-&-&-&-&-1.05&-1.13&\\&&&&&(0.58)&(0.58)&\\& \multicolumn{7}{c}{$\phi_{f}$: Father's Time}\\\cmidrule(r){2-8}Const.&2.31&2.38&1.93&3.76&2.73&3.29&\\&(1.71)&(1.89)&(1.60)&(2.87)&(2.39)&(2.73)&\\Father: College+&-0.53&-0.58&-0.74&-1.08&-0.83&-0.89&\\&(0.47)&(0.49)&(0.55)&(0.71)&(0.68)&(0.68)&\\Father: Some College&0.31&0.20&0.22&0.01&0.20&0.01&\\&(0.39)&(0.40)&(0.46)&(0.53)&(0.56)&(0.53)&\\Child Age&0.01&0.02&0.07&0.05&0.04&0.05&\\&(0.08)&(0.09)&(0.09)&(0.12)&(0.11)&(0.12)&\\Num. Children 0-5&-0.83&-0.51&-0.59&-0.86&-0.69&-0.73&\\&(0.46)&(0.44)&(0.48)&(0.68)&(0.63)&(0.64)&\\& \multicolumn{7}{c}{$\phi_{g}$: Goods}\\\cmidrule(r){2-8}Married&-1.08&-0.70&0.14&1.92&-0.51&-0.01&\\&(1.50)&(1.72)&(1.87)&(3.14)&(2.15)&(2.50)&\\Single&-1.45&-1.28&-0.36&0.84&-1.27&-0.98&\\&(1.13)&(1.25)&(1.49)&(2.38)&(1.52)&(1.74)&\\Mother: Some College&0.18&0.09&-&-0.01&0.18&0.06&\\&(0.36)&(0.37)&&(0.50)&(0.60)&(0.49)&\\Mother: College+&0.92&0.70&-&1.07&1.12&1.06&\\&(0.52)&(0.50)&&(0.72)&(0.80)&(0.73)&\\Father: College+&-0.32&-0.26&-0.10&-0.34&-0.47&-0.36&\\&(0.42)&(0.44)&(0.48)&(0.60)&(0.61)&(0.59)&\\Father: Some College&0.33&0.29&0.75&0.35&0.38&0.26&\\&(0.44)&(0.45)&(0.54)&(0.59)&(0.61)&(0.58)&\\Child Age&0.18&0.15&0.19&0.22&0.21&0.21&\\&(0.07)&(0.07)&(0.08)&(0.11)&(0.11)&(0.11)&\\Num. Children 0-5&-1.05&-1.13&-1.23&-1.60&-1.61&-1.57&\\&(0.47)&(0.48)&(0.54)&(0.78)&(0.81)&(0.76)&\\Type 2&-&-&-1.34&-1.85&-&-&\\&&&(0.81)&(1.11)&&&\\Type 3&-&-&-1.19&-1.56&-&-&\\&&&(0.76)&(1.02)&&&\\$\mu_{k}$&-&-&-&-&0.08&-0.04&\\&&&&&(0.54)&(0.54)&\\\bottomrule\end{tabular}
        \captionsetup{width=0.7\textwidth}
        %\caption*{Summary of Specifications Goes Here}
    \end{center}
\end{table}

\newgeometry{top=0in,bottom=0in}
\begin{table}\footnotesize\caption{\label{res1a}GMM Estimation of Relative Demand System with Wage Instruments}
    \begin{center}
        \begin{tabular}{lcccccc}\\\toprule
&(1)&(2)&(3)&(4)&(5)&\\\cmidrule(r){2-7}$\rho$&-0.69&-0.87&-1.68&-0.99&-1.76&\\
&(0.37)&(0.49)&(1.19)&(0.60)&(1.29)&\\
$\gamma$&-0.29&-0.36&-0.31&-0.34&-0.30&\\
&(0.16)&(0.17)&(0.16)&(0.16)&(0.15)&\\
& \multicolumn{6}{c}{$\phi_{m}$: Mother's Time}\\\cmidrule(r){2-7}Constant&3.94&4.30&5.55&5.36&7.29&\\&(0.43)&(0.62)&(1.67)&(1.39)&(3.15)&\\Single&0.10&0.11&0.11&0.13&0.12&\\&(0.13)&(0.15)&(0.21)&(0.16)&(0.21)&\\Mother: Some College&0.02&-&-0.13&-&-0.11&\\&(0.15)&&(0.28)&&(0.28)&\\Mother: College+&-0.27&-&-0.76&-&-0.78&\\&(0.24)&&(0.65)&&(0.69)&\\Child Age&-0.21&-0.24&-0.34&-0.25&-0.35&\\&(0.05)&(0.06)&(0.15)&(0.08)&(0.16)&\\Num. Children 0-5&0.10&0.12&0.16&0.15&0.20&\\&(0.10)&(0.12)&(0.18)&(0.14)&(0.20)&\\Type 2&-&-0.22&-0.49&-&-&\\&&(0.23)&(0.46)&&&\\Type 3&-&-0.55&-1.08&-&-&\\&&(0.39)&(0.86)&&&\\$\mu_{k}$&-&-&-&-0.64&-1.16&\\&&&&(0.48)&(0.96)&\\& \multicolumn{6}{c}{$\phi_{f}$: Father's Time}\\\cmidrule(r){2-7}Constant&2.97&2.99&3.35&3.03&3.35&\\&(0.42)&(0.46)&(0.76)&(0.50)&(0.79)&\\Father: College+&-0.12&-0.17&-0.41&-0.19&-0.41&\\&(0.23)&(0.25)&(0.45)&(0.28)&(0.47)&\\Father: Some College&0.07&0.07&-0.22&0.04&-0.24&\\&(0.21)&(0.23)&(0.45)&(0.25)&(0.47)&\\Child Age&-0.14&-0.16&-0.24&-0.17&-0.25&\\&(0.05)&(0.07)&(0.14)&(0.08)&(0.15)&\\Num. Children 0-5&0.16&0.18&0.29&0.21&0.32&\\&(0.14)&(0.15)&(0.26)&(0.17)&(0.28)&\\& \multicolumn{6}{c}{$\phi_{Y}$: Childcare}\\\cmidrule(r){2-7}Constant&-1.48&-1.50&-1.46&-1.37&-1.32&\\&(0.22)&(0.28)&(0.28)&(0.39)&(0.40)&\\Single&0.60&0.61&0.64&0.62&0.64&\\&(0.14)&(0.15)&(0.14)&(0.15)&(0.14)&\\Mother: Some College&0.00&-&-0.05&-&-0.02&\\&(0.13)&&(0.13)&&(0.13)&\\Mother: College+&-0.22&-&-0.28&-&-0.24&\\&(0.13)&&(0.13)&&(0.13)&\\Father: College+&-0.04&-0.01&0.01&-0.01&-0.01&\\&(0.17)&(0.18)&(0.17)&(0.18)&(0.17)&\\Father: Some College&-0.46&-0.48&-0.41&-0.50&-0.44&\\&(0.15)&(0.16)&(0.16)&(0.16)&(0.16)&\\Child Age&-0.04&-0.04&-0.04&-0.04&-0.04&\\&(0.02)&(0.02)&(0.02)&(0.02)&(0.02)&\\Num. Children 0-5&0.10&0.10&0.08&0.11&0.09&\\&(0.08)&(0.09)&(0.08)&(0.08)&(0.08)&\\Type 2&-&0.04&0.01&-&-&\\&&(0.21)&(0.20)&&&\\Type 3&-&-0.08&-0.10&-&-&\\&&(0.21)&(0.21)&&&\\$\mu_{k}$&-&-&-&-0.09&-0.10&\\&&&&(0.16)&(0.16)&\\& \multicolumn{6}{c}{Residual Correlation Test}\\\cmidrule(r){2-7}p-value&0.89&0.90&0.88&0.90&0.88&\\
\bottomrule\end{tabular}
        \captionsetup{width=0.7\textwidth}
        %\caption*{Summary of Specifications Goes Here}
    \end{center}
\end{table}


% \begin{table}\footnotesize\caption{\label{res1}GMM Estimation of Relative Demand System}
%     \begin{center}
%         \begin{tabular}{lcccccc}\\\toprule
&(1)&(2)&(3)&(4)&(5)&\\\cmidrule(r){2-7}$\rho$&-1.59&-1.87&-2.52&-2.15&-3.12&\\
&(0.49)&(0.60)&(0.91)&(0.73)&(1.29)&\\
$\gamma$&-0.77&-0.86&-0.80&-0.96&-0.91&\\
&(0.28)&(0.28)&(0.29)&(0.32)&(0.33)&\\
& \multicolumn{6}{c}{$\phi_{m}$: Mother's Time}\\\cmidrule(r){2-7}Married&4.44&6.44&9.33&6.68&11.04&\\&(2.43)&(2.89)&(4.68)&(3.15)&(6.01)&\\Single&3.46&5.20&7.39&5.47&8.86&\\&(1.68)&(2.12)&(3.42)&(2.34)&(4.46)&\\Mother: Some College&-0.69&-&-1.04&-&-1.52&\\&(0.42)&&(0.63)&&(0.86)&\\Mother: College+&0.03&-&-0.03&-&-0.17&\\&(0.38)&&(0.54)&&(0.64)&\\Child Age&0.19&0.22&0.28&0.22&0.32&\\&(0.11)&(0.13)&(0.18)&(0.14)&(0.22)&\\Num. Children 0-5&-0.42&-0.41&-0.47&-0.50&-0.55&\\&(0.34)&(0.38)&(0.52)&(0.41)&(0.61)&\\Type 2&-&-2.15&-2.49&-&-&\\&&(0.97)&(1.27)&&&\\Type 3&-&-2.16&-2.62&-&-&\\&&(0.95)&(1.27)&&&\\$\mu_{k}$&-&-&-&-1.46&-2.27&\\&&&&(0.68)&(1.13)&\\& \multicolumn{6}{c}{$\phi_{f}$: Father's Time}\\\cmidrule(r){2-7}constant&2.00&1.44&3.29&1.51&3.69&\\&(2.31)&(2.10)&(3.47)&(2.25)&(4.11)&\\Father: College+&-0.79&-0.90&-1.48&-0.82&-1.52&\\&(0.53)&(0.58)&(0.81)&(0.64)&(0.97)&\\Father: Some College&-0.01&-0.01&-0.31&-0.03&-0.44&\\&(0.43)&(0.49)&(0.60)&(0.53)&(0.71)&\\Child Age&0.37&0.49&0.60&0.50&0.68&\\&(0.15)&(0.19)&(0.26)&(0.20)&(0.33)&\\Num. Children 0-5&0.11&0.19&0.24&0.13&0.18&\\&(0.47)&(0.52)&(0.70)&(0.56)&(0.81)&\\& \multicolumn{6}{c}{$\phi_{g}$: Goods}\\\cmidrule(r){2-7}Married&-1.49&-0.68&0.78&-1.97&-0.36&\\&(2.00)&(2.22)&(3.44)&(2.18)&(3.70)&\\Single&-3.02&-2.59&-2.14&-3.94&-3.66&\\&(1.33)&(1.62)&(2.28)&(1.68)&(2.48)&\\Mother: Some College&-0.29&-&-0.37&-&-0.74&\\&(0.42)&&(0.61)&&(0.75)&\\Mother: College+&0.70&-&1.16&-&1.36&\\&(0.54)&&(0.82)&&(1.01)&\\Father: College+&-0.33&-0.28&-0.66&-0.27&-0.68&\\&(0.47)&(0.52)&(0.66)&(0.57)&(0.78)&\\Father: Some College&0.18&0.51&-0.03&0.60&-0.01&\\&(0.46)&(0.55)&(0.62)&(0.61)&(0.72)&\\Child Age&0.48&0.57&0.72&0.62&0.85&\\&(0.16)&(0.20)&(0.29)&(0.22)&(0.38)&\\Num. Children 0-5&-0.47&-0.61&-0.72&-0.71&-0.83&\\&(0.41)&(0.48)&(0.64)&(0.54)&(0.76)&\\Type 2&-&-1.52&-1.77&-&-&\\&&(0.94)&(1.22)&&&\\Type 3&-&-0.82&-0.87&-&-&\\&&(0.84)&(1.07)&&&\\$\mu_{k}$&-&-&-&0.08&0.04&\\&&&&(0.60)&(0.80)&\\& \multicolumn{6}{c}{Residual Correlation Test}\\\cmidrule(r){2-7}p-value&0.34&0.27&0.38&0.26&0.39&\\
\bottomrule\end{tabular}
%         \captionsetup{width=0.7\textwidth}
%         %\caption*{Summary of Specifications Goes Here}
%     \end{center}
% \end{table}

\begin{landscape}
    \begin{table}\footnotesize\caption{\label{res2}Joint GMM Estimation - Fully Restricted Case, No Binding Constraints}
        \begin{center}
            \begin{tabular}{lcccccccccccccccc}\\\toprule
 & \multicolumn{4}{c}{$\rho$} & \multicolumn{4}{c}{$\gamma$} & \multicolumn{4}{c}{$\delta_{1}$} & \multicolumn{4}{c}{$\delta_{2}$} \\
&(1)&(2)&(3)&(4)&(1)&(2)&(3)&(4)&(1)&(2)&(3)&(4)&(1)&(2)&(3)&(4)\\\cmidrule(r){2-5}\cmidrule(r){6-9}\cmidrule(r){10-13}\cmidrule(r){14-17}
&$-2.18$&$-2.85$&$-4.08$&$-4.52$&$-0.91$&$-1.25$&$-1.04$&$-1.02$&0.06&0.09&0.07&0.07&0.93&0.93&0.93&0.94\\
&(0.49)&(0.72)&(1.33)&(1.61)&(0.31)&(0.39)&(0.35)&(0.35)&(0.04)&(0.04)&(0.04)&(0.04)&(0.01)&(0.01)&(0.01)&(0.01)\\
&&&&&&&&&&&&&&&&\\
 & \multicolumn{4}{c}{$\phi_{m}$: Mother's Time} & \multicolumn{4}{c}{$\phi_{f}$: Father's Time} & \multicolumn{4}{c}{$\phi_{Y}$: Childcare} & \multicolumn{4}{c}{$\phi_{\theta}$: TFP} \\
&(1)&(2)&(3)&(4)&(1)&(2)&(3)&(4)&(1)&(2)&(3)&(4)&(1)&(2)&(3)&(4)\\\cmidrule(r){2-5}\cmidrule(r){6-9}\cmidrule(r){10-13}\cmidrule(r){14-17}
Const.&$5.06$&$6.32$&$8.31$&$13.13$&$3.26^{*}$&$3.47$&$4.07$&$4.16$&$-1.17^{**}$&$-1.22^{*}$&$-1.19^{*}$&$-1.44^{**}$&-0.80&-1.20&-1.04&-0.49\\
&(0.68)&(1.02)&(1.96)&(3.94)&(0.74)&(0.90)&(1.26)&(1.40)&(0.32)&(0.44)&(0.41)&(0.61)&(0.46)&(0.50)&(0.40)&(0.30)\\
Single&$0.12$&$0.10^{*}$&$-0.01^{*}$&$0.06^{*}$&-&-&-&-&$0.51^{**}$&$0.51$&$0.57$&$0.60^{*}$&-0.07&-0.07&-0.06&-0.06\\
&(0.24)&(0.28)&(0.37)&(0.41)&-&-&-&-&(0.20)&(0.24)&(0.21)&(0.21)&(0.06)&(0.06)&(0.06)&(0.06)\\
Mother: Some College&$-0.20$&-&$-0.34$&$-0.39$&-&-&-&-&$0.03$&-&$-0.01$&$0.04$&0.06&-&0.03&0.04\\
&(0.27)&-&(0.44)&(0.49)&-&-&-&-&(0.18)&-&(0.20)&(0.20)&(0.06)&-&(0.06)&(0.06)\\
Mother: College+&$-0.84$&-&$-1.60$&$-1.73$&-&-&-&-&$-0.22$&-&$-0.27$&$-0.23$&0.05&-&0.02&0.01\\
&(0.36)&-&(0.72)&(0.83)&-&-&-&-&(0.18)&-&(0.19)&(0.19)&(0.09)&-&(0.10)&(0.10)\\
Child Age&$-0.36$&$-0.44$&$-0.58$&$-0.64$&$-0.27^{**}$&$-0.34^{*}$&$-0.47^{*}$&$-0.51^{*}$&$-0.06^{**}$&$-0.06^{*}$&$-0.06$&$-0.06^{*}$&-0.02&-0.02&-0.02&-0.03\\
&(0.07)&(0.10)&(0.17)&(0.21)&(0.08)&(0.11)&(0.17)&(0.20)&(0.03)&(0.03)&(0.03)&(0.03)&(0.01)&(0.01)&(0.01)&(0.01)\\
Num. Children 0-5&$0.32$&$0.41$&$0.49$&$0.64$&$0.39$&$0.43$&$0.62$&$0.79$&$0.10$&$0.13$&$0.10$&$0.10$&0.14&0.16&0.15&0.14\\
&(0.18)&(0.23)&(0.30)&(0.36)&(0.24)&(0.29)&(0.41)&(0.48)&(0.12)&(0.14)&(0.12)&(0.13)&(0.05)&(0.05)&(0.05)&(0.05)\\
Type 2&-&$-0.81$&$-1.26$&-&-&-&-&-&-&$0.12$&$0.08$&-&-&0.22&0.23&-\\
&-&(0.41)&(0.61)&-&-&-&-&-&-&(0.34)&(0.31)&-&-&(0.08)&(0.08)&-\\
Type 3&-&$-1.96$&$-2.80^{*}$&-&-&-&-&-&-&$0.07$&$0.02$&-&-&0.01&0.02&-\\
&-&(0.62)&(1.03)&-&-&-&-&-&-&(0.34)&(0.31)&-&-&(0.12)&(0.12)&-\\
$\mu_{k}$&-&-&-&$-3.00^{*}$&-&-&-&-&-&-&-&$0.12^{**}$&-&-&-&-0.17\\
&-&-&-&(1.21)&-&-&-&-&-&-&-&(0.25)&-&-&-&(0.14)\\
Father: College+&-&-&-&-&$-0.47$&$-0.63$&$-1.05$&$-1.11$&$0.00$&$0.06$&$0.01$&$-0.03$&0.13&0.12&0.09&0.10\\
&-&-&-&-&(0.41)&(0.50)&(0.73)&(0.82)&(0.24)&(0.29)&(0.25)&(0.25)&(0.08)&(0.08)&(0.08)&(0.08)\\
Father: Some College&-&-&-&-&$-0.17$&$-0.22$&$-0.71$&$-0.84$&$-0.67$&$-0.71$&$-0.67$&$-0.73$&0.32&0.32&0.29&0.31\\
&-&-&-&-&(0.36)&(0.42)&(0.64)&(0.72)&(0.21)&(0.25)&(0.24)&(0.24)&(0.08)&(0.08)&(0.08)&(0.08)\\
ind02&-&-&-&-&-&-&-&-&-&-&-&-&0.14&0.16&0.14&0.15\\
&-&-&-&-&-&-&-&-&-&-&-&-&(0.06)&(0.06)&(0.06)&(0.06)\\
\\
\bottomrule\end{tabular}
            \captionsetup{width=1.7\textwidth}
            \caption*{Note: Superscripts indicate results of Lagrange Multiplier test of individual parameter restrictions. Rejection for a test of size 5\%, 1\% and 0.1\% is indicated by $^{*}$, $^{**}$, and $^{***}$.}
        \end{center}
    \end{table}        

    \begin{table}\footnotesize\caption{\label{res3}Joint GMM Estimation - Fully Restricted Case, No Borrowing or Saving}
        \begin{center}
            \begin{tabular}{lcccccccccccccccc}\\\toprule
 & \multicolumn{4}{c}{$\epsilon_{\tau,g}$} & \multicolumn{4}{c}{$\epsilon_{x,H}$} & \multicolumn{4}{c}{$\delta_{1}$} & \multicolumn{4}{c}{$\delta_{2}$} \\
&(1)&(2)&(3)&(4)&(1)&(2)&(3)&(4)&(1)&(2)&(3)&(4)&(1)&(2)&(3)&(4)\\\cmidrule(r){2-5}\cmidrule(r){6-9}\cmidrule(r){10-13}\cmidrule(r){14-17}
&$0.31$&$0.26$&$0.20$&$0.18$&$0.52$&$0.45$&$0.49$&$0.50$&0.13&0.14&0.12&0.13&0.92&0.92&0.93&0.93\\
&(0.05)&(0.05)&(0.05)&(0.05)&(0.08)&(0.08)&(0.08)&(0.08)&(0.05)&(0.05)&(0.04)&(0.04)&(0.01)&(0.01)&(0.01)&(0.01)\\
&&&&&&&&&&&&&&&&\\
 & \multicolumn{4}{c}{$\tilde{\phi}_{m}$: Mother's Time} & \multicolumn{4}{c}{$\tilde{\phi}_{f}$: Father's Time} & \multicolumn{4}{c}{$\tilde{\phi}_{x}$: Childcare} & \multicolumn{4}{c}{$\phi_{\theta}$: TFP} \\
&(1)&(2)&(3)&(4)&(1)&(2)&(3)&(4)&(1)&(2)&(3)&(4)&(1)&(2)&(3)&(4)\\\cmidrule(r){2-5}\cmidrule(r){6-9}\cmidrule(r){10-13}\cmidrule(r){14-17}
Constant&$5.15$&$6.27$&$8.33$&$13.23$&$3.28$&$3.40$&$4.03$&$4.12$&$-1.17^{+}$&$-1.20$&$-1.19$&$-1.43$&-1.53&-1.74&-1.45&-0.68\\
&(0.70)&(0.99)&(1.95)&(3.99)&(0.76)&(0.89)&(1.25)&(1.41)&(0.32)&(0.44)&(0.41)&(0.61)&(0.51)&(0.53)&(0.43)&(0.32)\\
Single&$0.14$&$0.12$&$0.02$&$0.10$&-&-&-&-&$0.51^{+}$&$0.50$&$0.56$&$0.59$&-0.11&-0.11&-0.10&-0.08\\
&(0.24)&(0.28)&(0.37)&(0.41)&-&-&-&-&(0.20)&(0.24)&(0.21)&(0.21)&(0.06)&(0.07)&(0.07)&(0.07)\\
Mother some coll.&$-0.19$&-&$-0.32$&$-0.38$&-&-&-&-&$0.04$&-&$-0.00$&$0.04$&0.01&-&-0.02&-0.01\\
&(0.28)&-&(0.44)&(0.50)&-&-&-&-&(0.19)&-&(0.20)&(0.20)&(0.07)&-&(0.07)&(0.07)\\
Mother coll+&$-0.85$&-&$-1.59$&$-1.74$&-&-&-&-&$-0.22$&-&$-0.26$&$-0.22$&-0.05&-&-0.09&-0.11\\
&(0.36)&-&(0.72)&(0.84)&-&-&-&-&(0.18)&-&(0.19)&(0.19)&(0.10)&-&(0.11)&(0.11)\\
Child's age&$-0.38$&$-0.44$&$-0.58$&$-0.66$&$-0.27$&$-0.33$&$-0.46$&$-0.51$&$-0.06^{+}$&$-0.07$&$-0.06$&$-0.06$&-0.01&-0.01&-0.02&-0.03\\
&(0.08)&(0.10)&(0.17)&(0.22)&(0.09)&(0.11)&(0.17)&(0.21)&(0.03)&(0.03)&(0.03)&(0.03)&(0.02)&(0.02)&(0.02)&(0.02)\\
Num. of children 0-5&$0.32$&$0.40$&$0.48$&$0.61$&$0.40$&$0.44$&$0.62$&$0.78$&$0.11$&$0.12$&$0.10$&$0.10$&0.13&0.14&0.14&0.12\\
&(0.19)&(0.22)&(0.30)&(0.36)&(0.25)&(0.29)&(0.42)&(0.49)&(0.12)&(0.14)&(0.12)&(0.13)&(0.05)&(0.05)&(0.05)&(0.05)\\
Type 2&-&$-0.79$&$-1.24$&-&-&-&-&-&-&$0.14$&$0.10$&-&-&0.13&0.14&-\\
&-&(0.40)&(0.61)&-&-&-&-&-&-&(0.34)&(0.31)&-&-&(0.10)&(0.10)&-\\
Type 3&-&$-1.91$&$-2.78$&-&-&-&-&-&-&$0.09$&$0.04$&-&-&-0.17&-0.16&-\\
&-&(0.60)&(1.02)&-&-&-&-&-&-&(0.34)&(0.32)&-&-&(0.15)&(0.15)&-\\
$\mu_{k}$&-&-&-&$-3.01$&-&-&-&-&-&-&-&$0.12^{+}$&-&-&-&-0.38\\
&-&-&-&(1.22)&-&-&-&-&-&-&-&(0.25)&-&-&-&(0.17)\\
Father some coll.&-&-&-&-&$-0.50$&$-0.61$&$-1.05$&$-1.14$&$-0.00$&$0.04$&$-0.01$&$-0.03$&0.09&0.06&0.07&0.08\\
&-&-&-&-&(0.42)&(0.49)&(0.73)&(0.83)&(0.24)&(0.28)&(0.25)&(0.25)&(0.08)&(0.09)&(0.08)&(0.08)\\
Father coll+&-&-&-&-&$-0.19$&$-0.20$&$-0.68$&$-0.83$&$-0.67$&$-0.72$&$-0.68$&$-0.74$&0.28&0.28&0.26&0.27\\
&-&-&-&-&(0.37)&(0.41)&(0.64)&(0.73)&(0.21)&(0.25)&(0.24)&(0.24)&(0.09)&(0.09)&(0.09)&(0.09)\\
Year = 2002&-&-&-&-&-&-&-&-&-&-&-&-&0.21&0.22&0.20&0.21\\
&-&-&-&-&-&-&-&-&-&-&-&-&(0.07)&(0.07)&(0.07)&(0.07)\\
\\
\bottomrule\end{tabular}
            \captionsetup{width=1.7\textwidth}
            \caption*{Note: Superscripts indicate results of Lagrange Multiplier test of individual parameter restrictions. Rejection for a test of size 5\%, 1\% and 0.1\% is indicated by $^{*}$, $^{**}$, and $^{***}$.}
        \end{center}
    \end{table}        


    \begin{table}\footnotesize\caption{\label{res4}Joint GMM Estimation - Unrestricted, No Binding Constraints}
        \begin{center}
            \begin{tabular}{lccccccc}\toprule
 & \multicolumn{2}{c}{$\rho$} & \multicolumn{2}{c}{$\gamma$} & {$\delta_{1}$} & {$\delta_{2}$} & $2N(Q_{N} - \tilde{Q}_{N})$ \\
 & (R) & (U) & (R) & (U) & - & - & - \\\cmidrule(r){2-3}\cmidrule(r){4-5}\cmidrule(r){6-6}\cmidrule(r){7-7}\cmidrule(r){8-8}
&-4.49& - &-1.01& - &0.11&0.92&3.40\\
&(1.54)& - &(0.34)& - &(0.05)&(0.02)&(0.49)\\
\\
&&&&&&&\\
 & \multicolumn{2}{c}{$\phi_{m}$: Mother's Time} & \multicolumn{2}{c}{$\phi_{f}$: Father's Time} & \multicolumn{2}{c}{$\phi_{Y}$: Childcare} &{$\phi_{\theta}$: TFP} \\
 & (R) & (U) & (R) & (U) & (R) & (U) & -  \\\cmidrule(r){2-3}\cmidrule(r){4-5}\cmidrule(r){6-7}\cmidrule(r){8-8}
Const.&9.02& -&4.29& -&-1.18&-0.66&-1.57\\
&(2.29)&&(1.40)&&(0.41)&(10.09)&(0.73)\\
Single&-0.00&4.06& - & -&0.58& -&0.06\\
&(0.40)&(21.96) & &&(0.21)&&(0.32)\\
Type 2&-1.38& -& - & -&0.08& -&0.22\\
&(0.68)& & &&(0.30)&&(0.17)\\
Type 3&-3.07&0.16& - & -&0.02& -&0.14\\
&(1.18)&(8.25) & &&(0.31)&&(0.47)\\
Mother: Some College&-0.39& -& - & -&-0.02& -&0.04\\
&(0.48)& & &&(0.19)&&(0.14)\\
Mother: College+&-1.78& -& - & -&-0.28& -&0.02\\
&(0.82)& & &&(0.19)&&(0.15)\\
Child Age&-0.65& -&-0.52&-0.71&-0.07& -&-0.01\\
&(0.20)&&(0.20)&(5.04)&(0.03)&&(0.06)\\
Num. Children 0-5&0.50& -&0.66& -&0.09& -&0.15\\
&(0.33)&&(0.46)&&(0.12)&&(0.07)\\
Father: College+& - & -&-1.17& -&0.01& -&0.07\\
 & &&(0.81)&&(0.25)&&(0.09)\\
Father: Some College& - & -&-0.82& -&-0.66& -&0.25\\
 & &&(0.71)&&(0.24)&&(0.23)\\
ind02& - & -& - & -& - & -&0.15\\
 & & & & & &&(0.08)\\
\\
\bottomrule\end{tabular}
            \captionsetup{width=\textwidth}
            \caption*{Note: the distance metric, $2N(Q_{N} - \tilde{Q}_{N})$, is the difference between the optimally weighted gmm criterion at the restricted estimates and its value at the relaxed estimates. It has a $\chi^{2}$ distribution with degrees of freedom equal to the number of constraints that are relaxed.  Standard errors are indicated in parentheses \emph{except} for the distance metric, which reports a p-value.}
        \end{center}
    \end{table}        

\end{landscape}


\end{document}





%%% Local Variables:
%%% mode: latex
%%% TeX-master: t
%%% End:
